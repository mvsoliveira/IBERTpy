% \documentclass{article}
% \usepackage{graphicx}
% \usepackage[a4paper, margin=0.5in]{geometry}
% \usepackage{subcaption}
% \usepackage{printlen}
% \uselengthunit{cm}

% \newlength\imageheight
% \newlength\imagewidth

% \begin{document}

\section{<title>}\label{sec:<id><Tex run>}

\begin{figure}[h] % "[t!]" placement specifier just for this example
\begin{subfigure}{0.5\textwidth}
\hyperref[sec:<fig00label><Tex run>]{\includegraphicsmaybe{<fig00>}}
\end{subfigure}\hspace*{\fill}
\begin{subfigure}{0.5\textwidth}
\hyperref[sec:<fig01label><Tex run>]{\includegraphicsmaybe{<fig01>}}
\end{subfigure}

\begin{subfigure}{0.5\textwidth}
\hyperref[sec:<fig02label><Tex run>]{\includegraphicsmaybe{<fig02>}}
\end{subfigure}\hspace*{\fill}
\begin{subfigure}{0.5\textwidth}
\hyperref[sec:<fig03label><Tex run>]{\includegraphicsmaybe{<fig03>}}
\end{subfigure}

\caption{<title>} \label{fig:<id><Tex run>}
\end{figure}

A cross-reference to Figure~\ref{fig:<id><Tex run>}.
<siblings> \\
Next summary Figure~\ref{fig:<id next><Tex run next>}.
\clearpage
% \end{document}